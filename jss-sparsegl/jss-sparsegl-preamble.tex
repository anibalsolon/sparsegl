\newcommand{\norm}[1]{\left\lVert #1 \right\rVert}
\newcommand{\snorm}[1]{\lVert #1 \rVert}
\newcommand{\R}{\mathbb{R}}
\newcommand{\vbeta}{\boldsymbol\beta}
\newcommand{\vx}{\mathbf{x}}
\newcommand{\mX}{\mathbf{X}}
\newcommand{\vY}{\mathbf{y}}
\newcommand{\vy}{\vY}
\newcommand{\df}{\mathrm{df}}
\newcommand{\vR}{\mathbf{r}}
\newcommand{\Hessian}{\mathbf{H}}
\usepackage{amsmath, amsfonts, amsthm, amssymb}
\usepackage{algorithm, algorithmic}
\usepackage{nicefrac}
\usepackage{booktabs}
\renewcommand{\algorithmiccomment}[1]{\hfill $\rhd$ #1}
\def\algorithmautorefname{Algorithm}
\def\T{\mathsf{T}}

\newtheorem{proposition}{Proposition}

\newlength\myindent % define a new length \myindent
\setlength\myindent{1em} % assign the length 1em to \myindet
\newcommand\bindent{%
  \begingroup % starts a group (to keep changes local)
  \setlength{\itemindent}{\myindent} % set itemindent (algorithmic internally uses a list) to the value of \mylength
  \addtolength{\algorithmicindent}{\myindent} % adds \mylength to the default indentation used by algorithmic
}
\newcommand\eindent{\endgroup} % closes a group
